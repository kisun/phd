\documentclass[11pt]{article}

\usepackage{amsmath}
\usepackage{amsfonts}
\usepackage{xfrac}

\usepackage{fullpage}
\usepackage{parskip}


\title{Kalman Filter}
\author{Overview of models}
\date{}

\begin{document}
\maketitle



\section{Simple discrete-time model}

For simplicity, we assume everything is discrete: positions are reported every $\delta_t$
seconds, and the input-control is specified using the stop time schedule, and assuming a
constant speed between stops. At a later date, we can fit more complex models that are
smooth up to the second derivative (i.e., acceleration).

The state vector is
\begin{equation}
  \label{eq:state-vector}
  \mathbf{x}_k = 
  \begin{bmatrix}
    x_k \\ v_k
  \end{bmatrix}
\end{equation}
where $x_k$ is the distance-into-trip (m) at time $k$, and $v_k$ is the velocity (m/s) at
time $k$.

The state-space equations are based on Newtonian physics, and use the schedule speed
information as input-control to adjust predictions accordingly.  The acceleration (i.e.,
change in speed) at time $k$ is defined as $a_k$, the change in velocity between the stops at
time $k$ and $k-\delta_t$.

Now, the state-space equations are:
\begin{align}
  \label{eq:state_space_eqns}
  x_k &= x_{k-1} + \left[v_{k-1} + a_k\right] \delta_t + \frac{\delta_t^2}{2} \varepsilon_k \\
  &= x_{k-1} + v_{k-1} \delta_t + a_k \delta_t + \frac{\delta_t^2}{2} \varepsilon_k \\
  \intertext{and}
  v_k &= v_{k-1} + a_k + \delta_t \varepsilon_a.
\end{align}
In the above, $\varepsilon_k \sim \mathcal{N}\left(0,\sigma_a^2\right)$ is random
variation in acceleration.

These equations can be expressed in vector notation as:
\begin{align}
  \label{eq:state_matrix_eqns}
  \mathbf{x}_k &= \mathbf{F} \mathbf{x}_{k-1} + \mathbf{B} a_k + \mathbf{G} a_k
\end{align}
where
\begin{align}
  \mathbf{F} = 
  \begin{bmatrix}
    1 & \delta_t \\
    0 & 1
  \end{bmatrix},
  \quad
  \mathbf{B} =
  \begin{bmatrix}
    \delta_t \\ 1
  \end{bmatrix}
  \quad\text{and}\quad
  \mathbf{G} = 
  \begin{bmatrix}
    \frac{\delta_t^2}{2} \\ \delta_t
  \end{bmatrix}.
\end{align}


We can re-express the state equations as
\begin{align}
  \label{eq:state_matrix_eqns2}
  \mathbf{x}_k &= \mathbf{F} \mathbf{x}_{k-1} + \mathbf{B} a_k + \mathbf{w}_k,
\end{align}
where $\mathbf{w}_k \sim \mathcal{N}\left(0,\mathbf{Q}\right)$, and 
\begin{equation}
  \label{eq:systm_variance}
  \mathbf{Q} = \mathbf{G} \mathbf{G}^T \sigma_a^2
  =
  \begin{bmatrix}
    \frac{\delta_t^2}{4} & \frac{\delta_t^3}{2} \\
    \frac{\delta_t^3}{2} & \delta_t^4
  \end{bmatrix} \sigma_a^2.
\end{equation}



The next step involves the noisy measurements of the location (distance-into-trip);
\begin{equation}
  \label{eq:measurements}
  \mathbf{z}_k = \mathbf{H}\mathbf{x}_k + \mathbf{v}_k,
\end{equation}
where the random measurment error is $\mathbf{v}_k \sim \mathcal{N}\left(0, \mathbf{R}\right)$,
\begin{equation}
  \label{eq:moremats}
  H = 
  \begin{bmatrix}
    0 & 1
  \end{bmatrix}
  \quad \text{and} \quad \mathbf{R} = \mathbb{E}\left[\mathbf{v}_k \mathbf{v}_k^T\right] = \sigma_z^2.
\end{equation}


We can now apply the Kalman Filter prediction and update equations to estimate the future.


\end{document}
