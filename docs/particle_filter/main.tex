\documentclass[11pt]{article}\usepackage[]{graphicx}\usepackage[]{color}
%% maxwidth is the original width if it is less than linewidth
%% otherwise use linewidth (to make sure the graphics do not exceed the margin)
\makeatletter
\def\maxwidth{ %
  \ifdim\Gin@nat@width>\linewidth
    \linewidth
  \else
    \Gin@nat@width
  \fi
}
\makeatother

\definecolor{fgcolor}{rgb}{0.345, 0.345, 0.345}
\newcommand{\hlnum}[1]{\textcolor[rgb]{0.686,0.059,0.569}{#1}}%
\newcommand{\hlstr}[1]{\textcolor[rgb]{0.192,0.494,0.8}{#1}}%
\newcommand{\hlcom}[1]{\textcolor[rgb]{0.678,0.584,0.686}{\textit{#1}}}%
\newcommand{\hlopt}[1]{\textcolor[rgb]{0,0,0}{#1}}%
\newcommand{\hlstd}[1]{\textcolor[rgb]{0.345,0.345,0.345}{#1}}%
\newcommand{\hlkwa}[1]{\textcolor[rgb]{0.161,0.373,0.58}{\textbf{#1}}}%
\newcommand{\hlkwb}[1]{\textcolor[rgb]{0.69,0.353,0.396}{#1}}%
\newcommand{\hlkwc}[1]{\textcolor[rgb]{0.333,0.667,0.333}{#1}}%
\newcommand{\hlkwd}[1]{\textcolor[rgb]{0.737,0.353,0.396}{\textbf{#1}}}%

\usepackage{framed}
\makeatletter
\newenvironment{kframe}{%
 \def\at@end@of@kframe{}%
 \ifinner\ifhmode%
  \def\at@end@of@kframe{\end{minipage}}%
  \begin{minipage}{\columnwidth}%
 \fi\fi%
 \def\FrameCommand##1{\hskip\@totalleftmargin \hskip-\fboxsep
 \colorbox{shadecolor}{##1}\hskip-\fboxsep
     % There is no \\@totalrightmargin, so:
     \hskip-\linewidth \hskip-\@totalleftmargin \hskip\columnwidth}%
 \MakeFramed {\advance\hsize-\width
   \@totalleftmargin\z@ \linewidth\hsize
   \@setminipage}}%
 {\par\unskip\endMakeFramed%
 \at@end@of@kframe}
\makeatother

\definecolor{shadecolor}{rgb}{.97, .97, .97}
\definecolor{messagecolor}{rgb}{0, 0, 0}
\definecolor{warningcolor}{rgb}{1, 0, 1}
\definecolor{errorcolor}{rgb}{1, 0, 0}
\newenvironment{knitrout}{}{} % an empty environment to be redefined in TeX

\usepackage{alltt}

\usepackage{amsmath}
\usepackage{amsfonts}
\usepackage{xfrac}

\usepackage{fullpage}
\usepackage{parskip}




\newcommand{\bx}{\boldsymbol{x}}
\newcommand{\bu}{\boldsymbol{u}}
\newcommand{\bw}{\boldsymbol{w}}
\newcommand{\bz}{\boldsymbol{z}}
\newcommand{\by}{\boldsymbol{y}}
\newcommand{\bh}{\boldsymbol{h}}
\newcommand{\bv}{\boldsymbol{v}}
\newcommand{\bfn}{\boldsymbol{f}}
\newcommand{\bF}{\boldsymbol{F}}
\newcommand{\bH}{\boldsymbol{H}}
\newcommand{\bK}{\boldsymbol{K}}
\newcommand{\bQ}{\boldsymbol{Q}}
\newcommand{\bR}{\boldsymbol{R}}
\newcommand{\bP}{\boldsymbol{P}}
\newcommand{\bS}{\boldsymbol{S}}
\newcommand{\bZero}{\boldsymbol{0}}
\newcommand{\dd}[2]{\frac{\partial {#1}}{\partial {#2}}}

\newcommand{\X}{\mathrm{X}}
\newcommand{\E}{\mathrm{E}}
\newcommand{\V}{\mathrm{V}}


\newcommand{\pr}{\mathbb{P}}
\renewcommand{\Pr}[1]{\pr\left(#1\right)}


\newcommand{\km}{_{k-1}}
\newcommand{\kk}{_{k|k}}
\newcommand{\kkm}{_{k|k-1}}
\newcommand{\kmkm}{_{k-1|k-1}}


\title{Particle Filter}
\author{Exploration}
\date{}
\IfFileExists{upquote.sty}{\usepackage{upquote}}{}
\begin{document}
\maketitle


\section{Straight Trajectory, Unconstrained}

An example in two-dimensions where the object travels in a straight trajectory. 


\begin{knitrout}
\definecolor{shadecolor}{rgb}{0.969, 0.969, 0.969}\color{fgcolor}\begin{kframe}
\begin{alltt}
\hlstd{x} \hlkwb{<-} \hlkwd{c}\hlstd{(}\hlnum{0}\hlstd{,} \hlnum{9}\hlstd{)}
\hlstd{y} \hlkwb{<-} \hlkwd{c}\hlstd{(}\hlnum{0}\hlstd{,} \hlnum{5}\hlstd{)}
\hlstd{t} \hlkwb{<-} \hlnum{1}
\hlkwd{plot}\hlstd{(x, y,} \hlkwc{type} \hlstd{=} \hlstr{"b"}\hlstd{,} \hlkwc{xlim} \hlstd{=} \hlkwd{c}\hlstd{(}\hlnum{0}\hlstd{,} \hlnum{20}\hlstd{),} \hlkwc{ylim} \hlstd{=} \hlkwd{c}\hlstd{(}\hlopt{-}\hlnum{20}\hlstd{,} \hlnum{20}\hlstd{),} \hlkwc{asp} \hlstd{=} \hlnum{1}\hlstd{,} \hlkwc{pch} \hlstd{=} \hlnum{19}\hlstd{)}
\hlkwd{abline}\hlstd{(}\hlkwc{h} \hlstd{=} \hlnum{0}\hlstd{,} \hlkwc{v} \hlstd{=} \hlnum{0}\hlstd{,} \hlkwc{lty} \hlstd{=} \hlnum{3}\hlstd{)}
\end{alltt}
\end{kframe}

{\centering \includegraphics[width=\maxwidth]{figure/first-1} 

}



\end{knitrout}

We can write the formula for movement as:
\begin{equation}
  \label{eq:straight_state}
  \X_k = \X_{k-1} + \V_{k-1}t_{k} + \E_kt_{k},
\end{equation}
where 
$\X_k = \left[x_k, y_k\right]^T$ are the coordinates at time $k$,
$\V_k = \left[\dot x_k, \dot y_k\right]^T$ are the $x$ and $y$ velocities at time $k$,
and $t_k$ is the time between steps $k-1$ and $k$; here, using uniform discrete time, $t_k = t$.
The error term is (for now) Gaussian zero-mean white-noise, $\E_k \sim \mathcal{N}\left(0, \Sigma_k\right)$,
$\Sigma_k = \Sigma =
\begin{bmatrix}
  \sigma_x^2 & 0 \\ 0 & \sigma_y^2
\end{bmatrix}$,
$\sigma_x^2 = 2$, $\sigma_y^2 = 3$.
For now, for velocity we will use $\dot x_k = \frac{x_k - x_{k-1}}{t_k}$, and likewise for $\dot y_k$.


So let us predict the next step, k = 2:
\begin{equation}
  \label{eq:predict}
  \hat \X_k = \X_{k-1} + \V_{k-1}t_k
\end{equation}
\begin{knitrout}
\definecolor{shadecolor}{rgb}{0.969, 0.969, 0.969}\color{fgcolor}\begin{kframe}
\begin{alltt}
\hlstd{X2} \hlkwb{<-} \hlkwd{rbind}\hlstd{(x[}\hlnum{2}\hlstd{], y[}\hlnum{2}\hlstd{])}
\hlstd{V2} \hlkwb{<-} \hlkwd{rbind}\hlstd{(x[}\hlnum{2}\hlstd{]} \hlopt{-} \hlstd{x[}\hlnum{1}\hlstd{], y[}\hlnum{2}\hlstd{]} \hlopt{-} \hlstd{y[}\hlnum{1}\hlstd{])} \hlcom{## t = 1}
\hlstd{Sigma} \hlkwb{<-} \hlkwd{cbind}\hlstd{(}\hlkwd{c}\hlstd{(}\hlnum{2}\hlstd{,} \hlnum{0}\hlstd{),} \hlkwd{c}\hlstd{(}\hlnum{0}\hlstd{,} \hlnum{3}\hlstd{))}

\hlstd{(X3hat} \hlkwb{<-} \hlstd{X2} \hlopt{+} \hlstd{V2} \hlopt \hlstd{t)}
\end{alltt}
\begin{verbatim}
##      [,1]
## [1,]   18
## [2,]   10
\end{verbatim}
\begin{alltt}
\hlkwd{plot}\hlstd{(x, y,} \hlkwc{type} \hlstd{=} \hlstr{"b"}\hlstd{,} \hlkwc{xlim} \hlstd{=} \hlkwd{c}\hlstd{(}\hlnum{0}\hlstd{,} \hlnum{20}\hlstd{),} \hlkwc{ylim} \hlstd{=} \hlkwd{c}\hlstd{(}\hlopt{-}\hlnum{20}\hlstd{,} \hlnum{20}\hlstd{),} \hlkwc{asp} \hlstd{=} \hlnum{1}\hlstd{,} \hlkwc{pch} \hlstd{=} \hlnum{19}\hlstd{)}
\hlkwd{abline}\hlstd{(}\hlkwc{h} \hlstd{=} \hlnum{0}\hlstd{,} \hlkwc{v} \hlstd{=} \hlnum{0}\hlstd{,} \hlkwc{lty} \hlstd{=} \hlnum{3}\hlstd{)}
\hlkwd{points}\hlstd{(X3hat[}\hlnum{1}\hlstd{], X3hat[}\hlnum{2}\hlstd{],} \hlkwc{pch} \hlstd{=} \hlnum{3}\hlstd{,} \hlkwc{col} \hlstd{=} \hlstr{"red"}\hlstd{)}
\end{alltt}
\end{kframe}

{\centering \includegraphics[width=\maxwidth]{figure/step_2-1} 

}



\end{knitrout}



That was easy enough! Now do it with particles\ldots

Require prior distribution $p\left(\X_0\right) \sim \mathcal{N}\left(\bZero, \Sigma_0\right)$,
$\Sigma_0 =
\begin{bmatrix}
  5 & 0 \\ 0 & 5
\end{bmatrix}$.

Step 1: sample the particles:
\begin{knitrout}
\definecolor{shadecolor}{rgb}{0.969, 0.969, 0.969}\color{fgcolor}\begin{kframe}
\begin{alltt}
\hlstd{N} \hlkwb{<-} \hlnum{500}  \hlcom{## for simplicity!!}
\hlstd{X} \hlkwb{<-} \hlkwd{rmvnorm}\hlstd{(N,} \hlkwc{sigma} \hlstd{=} \hlkwd{diag}\hlstd{(}\hlkwd{c}\hlstd{(}\hlnum{5}\hlstd{,} \hlnum{5}\hlstd{)))}

\hlkwd{plot}\hlstd{(x, y,} \hlkwc{type} \hlstd{=} \hlstr{"b"}\hlstd{,} \hlkwc{xlim} \hlstd{=} \hlkwd{c}\hlstd{(}\hlnum{0}\hlstd{,} \hlnum{20}\hlstd{),} \hlkwc{ylim} \hlstd{=} \hlkwd{c}\hlstd{(}\hlopt{-}\hlnum{20}\hlstd{,} \hlnum{20}\hlstd{),} \hlkwc{asp} \hlstd{=} \hlnum{1}\hlstd{,} \hlkwc{pch} \hlstd{=} \hlnum{19}\hlstd{)}
\hlkwd{abline}\hlstd{(}\hlkwc{h} \hlstd{=} \hlnum{0}\hlstd{,} \hlkwc{v} \hlstd{=} \hlnum{0}\hlstd{,} \hlkwc{lty} \hlstd{=} \hlnum{3}\hlstd{)}
\hlkwd{points}\hlstd{(X[,} \hlnum{1}\hlstd{], X[,} \hlnum{2}\hlstd{],} \hlkwc{pch} \hlstd{=} \hlnum{19}\hlstd{,} \hlkwc{cex} \hlstd{=} \hlnum{0.4}\hlstd{,} \hlkwc{col} \hlstd{=} \hlstr{"#00000030"}\hlstd{)}
\end{alltt}
\end{kframe}

{\centering \includegraphics[width=\maxwidth]{figure/sample_particles-1} 

}



\end{knitrout}

Let's assume we know the direction they'll be travelling:
\begin{knitrout}
\definecolor{shadecolor}{rgb}{0.969, 0.969, 0.969}\color{fgcolor}\begin{kframe}
\begin{alltt}
\hlstd{V} \hlkwb{<-} \hlstd{V2}
\hlstd{Xhats} \hlkwb{<-} \hlkwd{t}\hlstd{(}\hlkwd{t}\hlstd{(X)} \hlopt{+} \hlstd{V} \hlopt \hlkwd{rep}\hlstd{(t, N))}
\hlstd{Xprop} \hlkwb{<-} \hlstd{Xhats} \hlopt{+} \hlkwd{rmvnorm}\hlstd{(N,} \hlkwc{sigma} \hlstd{= Sigma)}

\hlkwd{plot}\hlstd{(x, y,} \hlkwc{type} \hlstd{=} \hlstr{"b"}\hlstd{,} \hlkwc{xlim} \hlstd{=} \hlkwd{c}\hlstd{(}\hlnum{0}\hlstd{,} \hlnum{20}\hlstd{),} \hlkwc{ylim} \hlstd{=} \hlkwd{c}\hlstd{(}\hlopt{-}\hlnum{20}\hlstd{,} \hlnum{20}\hlstd{),} \hlkwc{asp} \hlstd{=} \hlnum{1}\hlstd{,} \hlkwc{pch} \hlstd{=} \hlnum{19}\hlstd{)}
\hlkwd{abline}\hlstd{(}\hlkwc{h} \hlstd{=} \hlnum{0}\hlstd{,} \hlkwc{v} \hlstd{=} \hlnum{0}\hlstd{,} \hlkwc{lty} \hlstd{=} \hlnum{3}\hlstd{)}
\hlkwd{points}\hlstd{(Xprop[,} \hlnum{1}\hlstd{], Xprop[,} \hlnum{2}\hlstd{],} \hlkwc{pch} \hlstd{=} \hlnum{19}\hlstd{,} \hlkwc{cex} \hlstd{=} \hlnum{0.4}\hlstd{,} \hlkwc{col} \hlstd{=} \hlstr{"#0000FF30"}\hlstd{)}
\end{alltt}
\end{kframe}

{\centering \includegraphics[width=\maxwidth]{figure/particle_step1-1} 

}



\end{knitrout}

Now bootstrap by weights using 
\begin{equation}
  \label{eq:particle_weights}
  q_i = \frac{p\left(y_k | x_k^\star(i)\right)}{\sum_j p\left(y_k | x_k^\star(j)\right)},
  \quad
  p\left(y_k | x_k^\star\right) \sim \mathcal{N}\left(x_k^\star, R\right),
\end{equation}
where $R =
\begin{bmatrix}
  3 & 0 \\ 0 & 3
\end{bmatrix}$.

\begin{knitrout}
\definecolor{shadecolor}{rgb}{0.969, 0.969, 0.969}\color{fgcolor}\begin{kframe}
\begin{alltt}
\hlstd{R} \hlkwb{=} \hlkwd{diag}\hlstd{(}\hlkwd{c}\hlstd{(}\hlnum{3}\hlstd{,} \hlnum{3}\hlstd{))}
\hlstd{wi} \hlkwb{<-} \hlkwd{apply}\hlstd{(Xprop,} \hlnum{1}\hlstd{,} \hlkwa{function}\hlstd{(}\hlkwc{xi}\hlstd{)} \hlkwd{dmvnorm}\hlstd{(}\hlkwd{t}\hlstd{(X2), xi, R))}
\hlstd{qi} \hlkwb{<-} \hlstd{wi} \hlopt{/} \hlkwd{sum}\hlstd{(wi)}

\hlstd{i} \hlkwb{<-} \hlkwd{sample}\hlstd{(N, N,} \hlnum{TRUE}\hlstd{,} \hlkwc{prob} \hlstd{= qi)}
\hlstd{X} \hlkwb{<-} \hlstd{Xprop[i, ]}

\hlkwd{plot}\hlstd{(x, y,} \hlkwc{type} \hlstd{=} \hlstr{"b"}\hlstd{,} \hlkwc{xlim} \hlstd{=} \hlkwd{c}\hlstd{(}\hlnum{0}\hlstd{,} \hlnum{20}\hlstd{),} \hlkwc{ylim} \hlstd{=} \hlkwd{c}\hlstd{(}\hlopt{-}\hlnum{20}\hlstd{,} \hlnum{20}\hlstd{),} \hlkwc{asp} \hlstd{=} \hlnum{1}\hlstd{,} \hlkwc{pch} \hlstd{=} \hlnum{19}\hlstd{)}
\hlkwd{abline}\hlstd{(}\hlkwc{h} \hlstd{=} \hlnum{0}\hlstd{,} \hlkwc{v} \hlstd{=} \hlnum{0}\hlstd{,} \hlkwc{lty} \hlstd{=} \hlnum{3}\hlstd{)}
\hlkwd{points}\hlstd{(X[,} \hlnum{1}\hlstd{], X[,} \hlnum{2}\hlstd{],} \hlkwc{pch} \hlstd{=} \hlnum{19}\hlstd{,} \hlkwc{cex} \hlstd{=} \hlnum{0.4}\hlstd{,} \hlkwc{col} \hlstd{=} \hlstr{"#00000030"}\hlstd{)}
\end{alltt}
\end{kframe}

{\centering \includegraphics[width=\maxwidth]{figure/particle_weights-1} 

}



\end{knitrout}


\section{Straight Trajectory, Constrained to ``Road''}

We will now constrain the particles to the road.




\end{document}
