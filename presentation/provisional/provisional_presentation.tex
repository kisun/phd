\documentclass[10pt,t]{beamer}

\usetheme[progressbar=frametitle,sectionpage=none]{metropolis}

\usepackage{booktabs}
\usepackage[scale=2]{ccicons}

\usepackage{pgfplots}
\usepgfplotslibrary{dateplot}

\usepackage{xspace}
\newcommand{\themename}{\textbf{\textsc{metropolis}}\xspace}

\usepackage{tikz}
\usetikzlibrary{shapes.geometric, arrows}
\tikzset{font=\scriptsize}
\tikzstyle{startstop} = [rectangle, rounded corners, minimum width=2cm,%
minimum height=0.5cm, text centered, draw=black, fill=mLightBrown]
\tikzstyle{compute} = [rectangle, minimum width = 2cm, minimum height = 0.5cm,%
text centered, draw=black, fill=mLightGreen]
\tikzstyle{logic} = [diamond, minimum width = 1.2cm,%
text centered, draw=black, fill=mDarkTeal, text=white]

%% symbols
\newcommand{\bX}{\mathbf{X}}
\newcommand{\bY}{\mathbf{Y}}
\newcommand{\mat}[1]{\mathbf{#1}}
\renewcommand{\vec}[1]{\boldsymbol{#1}}

\title{Real time prediction of bus arrival}
%\subtitle{}
\date{July 31, 2016}
\author{Tom Elliott}
\institute{Supervised by Professor Thomas Lumley\\[2em]
\includegraphics[height=1.5cm]{stat-logo.png}}
%Department of Statistics\\University of Auckland}
%\titlegraphic{\hfill\includegraphics[height=1.5cm]{stat-logo.png}}

\begin{document}

\maketitle

\begin{frame}{Table of contents}
  \setbeamertemplate{section in toc}[sections numbered]
  \tableofcontents[hideallsubsections]
\end{frame}

\section{Introduction}

\begin{frame}[fragile]{Overview}
  \onslide<+->
  \begin{enumerate}[<+- | alert@+>]
    \item General Transit Feed Specification (GTFS)

    \item Sequential state space models
      \begin{itemize}[<1->]
        \item Kalman Filter
        \item Particle filter
      \end{itemize}

    \item Predictive models: future arrival times
      \begin{itemize}[<1->]
        \item Schedule
        \item Historical data
        \item Current state
        \item ``Real time'' travel/congestion from other buses
        \item Combinations of the above
      \end{itemize}

    \item Communicating predictive error to commuters

    \item Solving other common problems
  \end{enumerate}
  \onslide<+->
\end{frame}


\section{GTFS}

\begin{frame}{GTFS}
  \onslide<+->
  \begin{itemize}[<+- | alert@+>]
    \item General Transit Feed Specification
    \item GPS location of buses
    \item Static schedule information: trips, routes, shapes, schedules
    \item Standardised by Google and used globally
    \item API provided by Auckland Transport (AT)\\
      \url{https://dev-portal.at.govt.nz}
  \end{itemize}

  \begin{overprint}
    \onslide<5>
    \begin{figure}
      \vspace{-3em}
      \centering
      \includegraphics[height=4cm]{gtfs-feeds.png}
      \caption{Global GTFS feeds, \url{http://transitfeeds.com/}}
    \end{figure}
  \end{overprint}
  \onslide<+->
\end{frame}


\section{Modeling Real Time Data}

\begin{frame}{Modeling Real Time Data}
  \onslide<+->
  \begin{itemize}[<+- | alert@+>]
    \item new observations every 30 seconds:
      \begin{equation*}
        \bY_k =
        \begin{bmatrix}
          \phi_k & \lambda_k & t_k
        \end{bmatrix}^T
      \end{equation*}
      $\phi$ = latitude (north/south), $\lambda$ = longitude (east/west)

    \item each observation corresponds to an unknown \emph{state}:
      \begin{equation*}
        \bX_k =
        \begin{bmatrix}
          d_k & v_k & \cdots
        \end{bmatrix}^T
      \end{equation*}
      $d$ = distance into trip (m), $v$ = velocity (speed, m/s)

    \item want to \emph{update} the state of the model from the previous state using the latest observation
      \begin{equation*}
        \bX_k = f(\bX_{k-1}, \bY_k)
      \end{equation*}
  \end{itemize}
  \onslide<+->
\end{frame}

\begin{frame}{Modeling Real Time Data: Kalman Filter I}
  \onslide<+->
  \begin{itemize}[<+- | alert@+>]
    \item Used in robotics, GPS tracking, etc.

    \item Several bus tracking/prediction applications\\
      (Cathey \& Dailey, 2004; more people \ldots)

    \item Very fast (matrix multiplication), two steps:
      \begin{enumerate}
        \item Predict:
          \begin{equation*}
            \bX_k = \mat{A}_k \bX_{k-1} + \vec{w}_k
          \end{equation*}
          $\mat{A}_k$ depends on $\delta_k$ = time since last observation\\
          $\vec{w}_k$ = Gaussian process noise
        \item Update:
          \begin{equation*}
            \vec{z}_k = \mat{H} \bX_k + \vec{v}_k
          \end{equation*}
          $\mat{H} $ = observation model\\
          $\vec{v}_k$ = Gaussian measurement error\\
          $\vec{z}_k$ = observed data$^\dagger$
      \end{enumerate}
  \end{itemize}
  \onslide<+->
\end{frame}

\begin{frame}{Modeling Real Time Data: Kalman Filter II}
  \onslide<+->
  \begin{itemize}[<+- | alert@+>]
      \item Assumes all errors are normal

      \item Cannot map $\bX_k$ directly to $\bY_k$,
        complicated algorithms to first \emph{estimate} $\vec{z}_k$

      \item All dynamics need to be specified in $\mat{A}_k$\\
        e.g., $d_k = d_{k-1} + v_{k-1}\delta_k + \text{process noise}$
        (Newton's laws of motion)

      \item not easy to include complex features:
        e.g., dwell times

      \item can't cope with multi-modal posterior:
        \begin{itemize}[<1->]
          \item loops
          \item delays/detours
        \end{itemize}

      \item posterior is (multivariate) normal
  \end{itemize}
  \onslide<+->
\end{frame}


\section{Particle Filter}

\begin{frame}{Modeling Real Time Data: Particle Filter I}
  \onslide<+->
  \begin{itemize}[<+- | alert@+>]
    \item Generalises almost everything from KF

    \item Simulates ``imaginary'' buses with state $\bX_k^{(i)}$

    \item Predict each particle to generate \emph{prior distribution}:
      \begin{equation*}
        \tilde\bX_k^{(i)} = f(\bX_{k-1}^{(i)}, \sigma_v^2)
      \end{equation*}
      No restrictions on $f$ (except computational)

    \item Resample particles based on observed GPS coordinates:
      \begin{equation*}
        w_1^{(i)} = \frac{p(\bY_k | \tilde\bX_k^{(i)})}{\sum_{j=1}^M p(\bY_k | \tilde\bX_k^{(j)})}
      \end{equation*}
      $p(y,x)$ is the \emph{likelihood}

    \item New sample is the \emph{posterior distribution} of $\bX_k$
  \end{itemize}
  \onslide<+->
\end{frame}


\begin{frame}{Modeling Real Time Data: Particle Filter II}
  A simple example using only one dimension (distance).

  \onslide<+->

  \begin{itemize}[<+- | alert@+>]
    \item Start with previous state $\bX_{k-1}$
    \item Use model to predict state $\tilde\bX_{k}$
    \item Add process noise
    \item Get observation $\bY_k$
    \item Compute weights based on distance from observation
    \item Weighted resample to get $\bX_k$
  \end{itemize}
  %\vspace{5cm}
  \begin{overprint}
    \onslide<2>
    \centering
    \includegraphics[width=0.8\textwidth]{figs/pf1-frame1.png}
    \onslide<3>
    \centering
    \includegraphics[width=0.8\textwidth]{figs/pf1-frame2.png}
    \onslide<4>
    \centering
    \includegraphics[width=0.8\textwidth]{figs/pf1-frame3.png}
    \onslide<5>
    \centering
    \includegraphics[width=0.8\textwidth]{figs/pf1-frame4.png}
    \onslide<6>
    \centering
    \includegraphics[width=0.8\textwidth]{figs/pf1-frame5.png}
    \onslide<7->
    \centering
    \includegraphics[width=0.8\textwidth]{figs/pf1-frame6.png}
  \end{overprint}
  \onslide<+->
\end{frame}


\begin{frame}{Modeling Real Time Data: Model I}
  \onslide<+->
  \begin{itemize}[<+- | alert@+>]
    \item Newton's Laws of Motion
      \begin{itemize}[<1->]
        \item Model dynamic motion of vehicles with time\\
          $d_k = d_{k-1} + \delta_k v_{k}$, $v_k = v_{k-1} + \text{noise}$
        \item Not good for buses stopping frequently for passengers
      \end{itemize}

    \item Bus behaviour
      \begin{itemize}
        \item Follow motion laws between stops
        \item Stop at bus stop $j$ with probability $\pi_j$
        \item If bus stops:
          \begin{itemize}[<1->]
            \item decelerate and open doors
            \item allow passengers to alight and embark
            \item close doors and accelerate
          \end{itemize}
        \item If bus doesn't stop, do none of the above
        \item Traffic lights: just use a very small ``$\pi$''
          that works in almost the same way
      \end{itemize}
  \end{itemize}
  \onslide<+->
\end{frame}

\begin{frame}{Modeling Real Time Data: Model II}
  \onslide<+->
  \begin{itemize}[<+- | alert@+>]
    \item Dwell time at stop $j$
      \begin{itemize}
        \item Given a bus stops \ldots
        \item it will always decelerate,
          open/close doors, and accelerate: $\gamma$
        \item it will then wait at the bus stop until passengers
          have alighted/boarded: $\tau_j \sim \mathcal{E}(\mu_\tau)$
        \item Total travel time lost at bus stop = dwell time
      \end{itemize}

    \item Particle filter model
      \begin{itemize}
        \item Just include logic in $f$
        \item $p_j^{(i)} \sim \text{Bernoulli}(\pi_j)$
        \item if $p_j^{(i)} = 1$, then \ldots\\
          \ldots wait at stop for $\gamma + \tau_j^{(i)}$~seconds before continuing
      \end{itemize}
  \end{itemize}
  \onslide<+->
\end{frame}

\begin{frame}{Modeling Real Time Data: Model III}
  Particle Filter model, $X_k = f(X_{k-1}, Y_k, \sigma_v, \sigma_y)$

  \begin{tikzpicture}[node distance=1.5cm]
    \node (start) [startstop] {Previous State};
    \node (atStop) [logic, right of=start, xshift=1cm] {\tiny At stop?};
    \node (move1) [compute, right of=atStop, xshift=1.5cm] {Move Forward};
    \node (passStop) [logic, below of=move1] {\tiny Pass stop?};
    \node (arrival) [compute, below of=passStop,align=center] {Compute arrival time\\ and sample $p$};
    \node (dwell) [compute, left of=arrival,xshift=-1.5cm] {Sample dwell time};
    \node (continue) [logic, below of=dwell] {\tiny Time left?};
    \node (stateEst) [startstop, right of=continue, xshift=5cm] {Estimated State};
    \node (resample) [compute, below of=stateEst,yshift=0.7cm] {Weighted Resample};
    \node (stop) [startstop, below of=resample,yshift=0.7cm] {Final State};
  \end{tikzpicture}
\end{frame}


\begin{frame}[standout]
  Thank you!
\end{frame}


\end{document}
