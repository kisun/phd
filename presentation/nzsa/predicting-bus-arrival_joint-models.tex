\documentclass[10pt,t]{beamer}\usepackage[]{graphicx}\usepackage[]{color}
%% maxwidth is the original width if it is less than linewidth
%% otherwise use linewidth (to make sure the graphics do not exceed the margin)
\makeatletter
\def\maxwidth{ %
  \ifdim\Gin@nat@width>\linewidth
    \linewidth
  \else
    \Gin@nat@width
  \fi
}
\makeatother

\definecolor{fgcolor}{rgb}{0.345, 0.345, 0.345}
\newcommand{\hlnum}[1]{\textcolor[rgb]{0.686,0.059,0.569}{#1}}%
\newcommand{\hlstr}[1]{\textcolor[rgb]{0.192,0.494,0.8}{#1}}%
\newcommand{\hlcom}[1]{\textcolor[rgb]{0.678,0.584,0.686}{\textit{#1}}}%
\newcommand{\hlopt}[1]{\textcolor[rgb]{0,0,0}{#1}}%
\newcommand{\hlstd}[1]{\textcolor[rgb]{0.345,0.345,0.345}{#1}}%
\newcommand{\hlkwa}[1]{\textcolor[rgb]{0.161,0.373,0.58}{\textbf{#1}}}%
\newcommand{\hlkwb}[1]{\textcolor[rgb]{0.69,0.353,0.396}{#1}}%
\newcommand{\hlkwc}[1]{\textcolor[rgb]{0.333,0.667,0.333}{#1}}%
\newcommand{\hlkwd}[1]{\textcolor[rgb]{0.737,0.353,0.396}{\textbf{#1}}}%
\let\hlipl\hlkwb

\usepackage{framed}
\makeatletter
\newenvironment{kframe}{%
 \def\at@end@of@kframe{}%
 \ifinner\ifhmode%
  \def\at@end@of@kframe{\end{minipage}}%
  \begin{minipage}{\columnwidth}%
 \fi\fi%
 \def\FrameCommand##1{\hskip\@totalleftmargin \hskip-\fboxsep
 \colorbox{shadecolor}{##1}\hskip-\fboxsep
     % There is no \\@totalrightmargin, so:
     \hskip-\linewidth \hskip-\@totalleftmargin \hskip\columnwidth}%
 \MakeFramed {\advance\hsize-\width
   \@totalleftmargin\z@ \linewidth\hsize
   \@setminipage}}%
 {\par\unskip\endMakeFramed%
 \at@end@of@kframe}
\makeatother

\definecolor{shadecolor}{rgb}{.97, .97, .97}
\definecolor{messagecolor}{rgb}{0, 0, 0}
\definecolor{warningcolor}{rgb}{1, 0, 1}
\definecolor{errorcolor}{rgb}{1, 0, 0}
\newenvironment{knitrout}{}{} % an empty environment to be redefined in TeX

\usepackage{alltt}

\usetheme[progressbar=frametitle]{metropolis}

\usepackage{booktabs}
\usepackage{natbib}
\usepackage[scale=2]{ccicons}
\usepackage{lmodern}



%\usepackage{xspace}
%\newcommand{\themename}{\textbf{\textsc{metropolis}}\xspace}
%\renewcommand\textbullet{\ensuremath{\bullet}}

\newcommand{\bX}{\mathbf{X}}
\newcommand{\bY}{\mathbf{Y}}
\newcommand{\mat}[1]{\mathbf{#1}}
\renewcommand{\vec}[1]{\boldsymbol{#1}}


\title{Real-time prediction of bus arrival using joint models for vehicle and road}
%\subtitle{}
\author{Tom Elliott}
\date{}
\institute{Supervisor: Professor Thomas Lumley\\[2em]}
%\includegraphics[height=1.5cm,width=3.8cm]{stat-logo.png}}

%Department of Statistics\\University of Auckland}
%\titlegraphic{\hfill\includegraphics[height=1.5cm]{stat-logo.png}}
\IfFileExists{upquote.sty}{\usepackage{upquote}}{}
\begin{document}

\maketitle


\begin{frame}
  \frametitle{Overview}
  
  \begin{enumerate}
  \item A quick motivation
    
  \item Two recursive models: vehicle (particle filter) \& road (Kalman filter)
    
  \item Predicting arrival times
  \end{enumerate}
\end{frame}


\begin{frame}
  \frametitle{What's wrong with the current\footnote{Auckland Transport} system?}
  
  \begin{itemize}
  \item Prediction inaccuracy
    
  \item Prone to errors
    
  \item Statistical model?
  \end{itemize}
\end{frame}


\section{Vehicle State Model}

\begin{frame}
  \frametitle{Vehicle State Model}
  
  \textbf{Goal:} use observations of bus location (GPS) \ldots
  \begin{equation*}
    \bY_k = 
    \begin{bmatrix}
      \phi_k \\ \lambda_k \\ t_k
    \end{bmatrix} =
    \begin{bmatrix}
      \text{latitude (degrees)} \\
      \text{longitude (degrees)} \\
      \text{timestamp}
    \end{bmatrix}
  \end{equation*}
  \ldots to infer \textbf{unobservable vehicle state} \ldots
  \begin{equation*}
    \bX_k =
    \begin{bmatrix}
      d_k \\ v_k \\ s_k \\ \vdots
    \end{bmatrix} =
    \begin{bmatrix}
      \text{distance into trip (meters)} \\
      \text{velocity/speed ($ms^{-1}$)} \\
      \text{last visisted stop} \\
      \vdots
    \end{bmatrix}
  \end{equation*}
  \ldots in real time.
\end{frame}

\begin{frame}
  \frametitle{Vehicle State Model}
  
[image of data location] $\Rightarrow$
[images of vehicle state - distance into trip]

\end{frame}


\begin{frame}
  \frametitle{Vehicle State Model: Particle Filter}
  
  \begin{itemize}
  \item Flexible modeling framework, fewer assumptions
    
  \item Proposals cover more plausible states
    
  \item Easy to compute likelihood
  \end{itemize}
  
  
  \textbf{Step 1}: predict new state (for each particle independently)
  \begin{equation*}
    \bX_k^{(i)} = f(\bX_{k-1}^{(i)}, \sigma_v^2, \ldots)
  \end{equation*}
  {\small$f$: transition function; $\sigma_v^2$: process noise; shape information, \ldots}
  
  
  \textbf{Step 2}: resample particles based on likelihood
  \begin{equation*}
    \bY_k = h(\bX_k) + e_k
  \end{equation*}
  {\small$f$: measurement function; $e_k$ measurement (GPS) error}
\end{frame}


\begin{frame}
  \frametitle{Vehicle State Model: Particle Filter}
  
  
\end{frame}



\end{document}
