\documentclass[10pt,a4paper]{article}

\usepackage{fullpage}
\usepackage{amsmath}
\usepackage{amsfonts}

\title{Particle Filter applied to GTFS Realtime}
\author{Tom Elliott}
\date{2016}

\begin{document}

\maketitle


The Particle Filter (PF) model will allow buses to stop at bus stops
for some exponentially distributed time.
The speed component of the model will be fairly simple
(normally distributed, truncated between 0 and 30~ms$^{-1}$).
The arrival time of each particle will be recorded, 
as will the corresponding departure time. 
This should give a posterior sample of dwell times at each stop.


The condition for ``at a stop'' will be ``within 20~m of the bus stop'',
which allows for GPS error, as well as queues or other obstructions at stops.
Additionally, there will be a small probability that a bus doesn't move from its
previous location, even if its between stops (e.g., when stopped at traffic lights).



The overall state vector associated with observation $k$ is
\begin{equation}
  \label{eq:state_vector}
  X_k =
  \begin{bmatrix}
    d_k & v_k & s_k & T_{s_k} & D_{s_k}
  \end{bmatrix}^T
\end{equation}
where $d_k$ is the distance into trip (m),
$v_k$ is the speed (ms$^{-1}$),
$s_k$ is the most recently visisted stop number,
$T_j$ is the arrival time at stop $j$,
and $D_j$ the associated departure time.
The dwell time between arrival and departure at stop $j$ is $\tau_j$,
the the probability of stopping is $\pi_j$.
Each particle has its own associated state vector, denoted by a superscript $(i)$,
for example $X_k^{(i)}$.

The data are GPS coordinates and timestamps.
The elapsed time between observations is denoted $\delta_k = t_k - t_{k-1}$.
To compute likelihoods for the particles, and therefore perform weighted resampling,
the distances $d_k$ are transformed into GPS coordinates, using the shape associated
with the trip and interpolating.
The distance is thence just a great circle distance\footnote{In fact, 
the flat Earth approximation is within $1\times 10^{-15}$~m in our case, 
so it will suffice if computation times need reducing.}
between two points,
\begin{equation}
  \label{eq:greatcircle}
  \Delta d = R \arcsin(\sin(\phi_1) \cdot \sin(\phi_2) +
  \cos(\phi_1) \cdot \cos(\phi_2) \cdot \cos(|\lambda_1 - \lambda_2|))
\end{equation}
where $R$ is the Earth's radius is the desired units (we use $R = 6.371\times 10^6$~m).
The particle weights are based on the distances being normally distributed
with mean 0 and varince a model parameter to tune (although it is effectively GPS error).
Too small and too few particles will be resamples; too large and the accuracy will be low.




\end{document}
