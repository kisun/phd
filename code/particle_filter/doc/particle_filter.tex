\documentclass[14paper,twoside]{article}

\usepackage{amsmath}
\usepackage{amsfonts}
\usepackage{bbm}


\usepackage[hidelinks]{hyperref}
\usepackage{cleveref}

\usepackage{fullpage}


\newcommand{\bX}{\mathbf{X}}
\newcommand{\bY}{\mathbf{Y}}
\newcommand{\bZ}{\mathbf{Z}}
\newcommand{\bnu}{\boldsymbol{\nu}}
\newcommand{\bw}{\mathbf{w}}
\newcommand{\bQ}{\mathbf{Q}}
\newcommand{\bR}{\mathbf{R}}
\newcommand{\bz}{\mathbf{z}}
\newcommand{\bv}{\mathbf{v}}
\newcommand{\br}{\mathbf{r}}
\newcommand{\bS}{\mathbf{S}}
\newcommand{\bK}{\mathbf{K}}

\newcommand{\f}[1]{f\left(#1\right)}
\newcommand{\g}[1]{g\left(#1\right)}
\newcommand{\lhood}[1]{\ell\left(#1\right)}


\title{Real-time Bus Model Algorithm}
\author{Tom Elliott}
\date{}

\begin{document}

\maketitle

\section{Outline}
\label{sec:outline}

The algorithm runs in real-time\footnote{or at least in chronological sequence in the case of testing}
and ``simultaneously'' models the state of vehicles (\cref{sec:bus}),
and the state of route segments (\cref{sec:speed}).



\section{Vehicle State Model: Particle Filter}
\label{sec:bus}


The vehicle state is modeled using a particle filter, 
as this gives good coverage of all plausible states.
Assuming we are only interested in a single vehicle,
we have unknown state $\bX_k$ associated with observation $\bY_k$,
at time $t_k$.


The vehicle state is
\begin{equation}
  \label{eq:vehicle_state}
  \bX_k = 
  \begin{bmatrix}
    d_k \\ v_k \\ s_k \\ T_{s_k} \\ D_{s_k}
  \end{bmatrix} =
  \left[
    \begin{array}{l}
      \text{distance into trip (m)} \\
      \text{velocity (speed) of the vehicle } (ms^{-1}) \\
      \text{(route) segment identifier / last stop} \\
      \text{arrival time at the last stop} \\
      \text{departure time from the last stop}
    \end{array}
  \right].
\end{equation}
Within the framework of the particle filter, the state is represented by a sample of \emph{particles},
each with it's own state estimate, denoted by a superscript: $\bX_k^{(i)}$.

The observations are GPS coordinates plus a timestamp,
\begin{equation}
  \label{eq:vehicle_observation}
  \bY_k =
  \begin{bmatrix}
    \phi_k \\ \lambda_k \\ t_k
  \end{bmatrix} =
  \left[
    \begin{array}{l}
      \text{latitude (degrees north/south)} \\
      \text{longitude (degrees east/west)} \\
      \text{timestamp (seconds since 12am 01/01/1970)}
    \end{array}
  \right]
\end{equation}


To compare the state to the observation, 
we transform the particles states into a coordinate space using the function $h$.
This uses the shape information to transform the distance into trip measurment for particle $i$, $d_k^{(i)}$,
into a GPS coordinate $\bY_k^{(i)}$;
we then use an equirectangular projection 
$\left\{g : \bY_k^{(i)} \,|\, \bY_k \mapsto \bZ_k^{(i)} \right\}$ 
on the particles, centered on the observation, to obtain 
\begin{equation}
  \label{eq:transform}
  \g{\bY_k^{(i)} | \bY_k, \sigma^2_y} = \bZ_k^{(i)} =
  \begin{bmatrix}
    x_k^{(i)} \\ y_k^{(i)}
  \end{bmatrix} =
  \begin{bmatrix}
    \left( \Lambda_k^{(i)} - \Lambda_k \right) \cos \left( \Phi_k \right) \\
    \Phi_k^{(i)} - \Phi_k
  \end{bmatrix}
\end{equation}
where $\Phi = \frac{\pi}{180} \phi$ and $\Lambda = \frac{\pi}{180} \lambda$ 
(i.e., the latitude and longitude in radians).
Note that $g(\bY_k) = \left[0\ 0\right]^T$.
We can therefore use a bivariate normal to compute the likelihood of each particle
using $\boldsymbol{\mu} = \begin{bmatrix} 0 \\ 0 \end{bmatrix}$
and $\Sigma = \frac{\sigma^2_y}{R} I_2$ where $I_n$ is an $n\times n$ identity matrix:
\begin{align}
  \label{eq:particle_likelihood}
  \g{h(\bX_k^{(i)}) \,|\, \bY_k, \sigma^2_y} = \bZ_k^{(i)} 
  &\sim  \mathcal{N}\left(
    \begin{bmatrix}
      0 \\ 0
    \end{bmatrix},
    \frac{\sigma_y^2}{R}\cdot I_2
  \right) \\
  \lhood{\bX_k^{(i)}| \bY_k, \sigma^2_y} 
  &= \frac{1}{\sqrt{\left(2\pi\right)^2 \left| \boldsymbol{\Sigma} \right|}}
    e^{-\frac{1}{2}\left(g(h(\bX_k^{(i)})|\bY_k) - \boldsymbol{\mu}\right)^T\boldsymbol\Sigma^{-1}\left(g(h(\bX_k^{(i)})|\bY_k) - 
    \boldsymbol{\mu}\right)} \nonumber \\
  &\propto
    e^{-\frac{1}{2}  \left(g(h(\bX_k^{(i)})|\bY_k)\right)^T \left(\frac{\sigma^2_y}{R} I_2 \right)^{-1} \left(g(h(\bX_k^{(i)})|\bY_k)\right) } \nonumber \\
  \label{eq:particle_likelihood_final}
  \lhood{\bX_k^{(i)}| \bY_k, \sigma^2_y} 
  &\propto e^{-\frac{R}{2\sigma^2}\left((x_k^{(i)})^2 + (y_k^{(i)})^2\right)}
\end{align}
where $x_k^{(i)}$ and $y_k^{(i)}$ are as defined in \cref{eq:transform},
and we can drop the constant since it is common to all particles,
so is not required for computing weights.

The weight of each particle is simply computed as
\begin{equation}
  \label{eq:particle_weights}
  w_i = \frac{\lhood{\bX_k^{(i)} | \bY_k, \sigma^2_y}}{\sum_{j=1}^M \lhood{\bX_k^{(j)} | \bY_k, \sigma^2_y}}
  = \frac{e^{-\frac{R}{2\sigma^2}\left((x_k^{(i)})^2 + (y_k^{(i)})^2\right)}}{\sum_{j=1}^M e^{-\frac{R}{2\sigma^2}\left((x_k^{(j)})^2 + (y_k^{(j)})^2\right)}}.
\end{equation}


The last step is the transition function $\left\{ f : \bX_{k-1}^{(i)} \mapsto \bX_k^{(i)} \right \}$,
which effectively describes the behaviour of the vehicle.
There are several components we must consider:
\begin{enumerate}
\item is the bus at a stop? (or perhaps stopped at an intersection?)
\item will the bus pass one or more stops during the next time step?
\item if it does, does it stop?
\item if it stops, how long does it wait there?
\end{enumerate}

First we define $\delta_k = t_k - t_{k-1}$.

\subsection{Is the bus at a stop?}

At the start of each iteration, we first decide whether the bus still stay where it is,
and if so for how long.
The values used depend on whether the bus is at a stop or not (in which case it \emph{could} be 
at an intersection).

First off, if the bus \emph{is not} at a bus stop, we define $\rho$ as the probability that is
is currently stopped.
Then we define $\nu$ as the average time spent stopped due to, for example, traffic lights.
Thus, at the beginning of an iteration, we reduce the time available to travel $\tilde \delta_k$ 
by sampling $q_i \sim \mathrm{Bernoulli}(\rho)$ and $c_i \sim \mathcal{E}\left(\frac{1}{\upsilon}\right)$,
leaving us with
\begin{equation}
  \label{eq:start_intersection}
  \tilde\delta_k = 
  \begin{cases}
    \delta_k - q_i c_i & q_i c_i \leq \delta_k, \\
    0 & \text{otherwise}.
  \end{cases}
\end{equation}

Alternatively, if the bus \emph{is} at a stop, we detect this by checking if $A_{s_k}^{(i)}$ is set, 
and $D_{s_k}^{(i)}$ is not.
When the vehicle is at a stop, we must ensure that is stays for \emph{at least} $\gamma$ seconds
($\gamma$ is the time taken to decelerate, open doors, close doors, and accelerate, 
i.e., the minimum dwell time).
Then, since the exponential distribution is memoryless, we simply draw a new dwell time for each particle,
$c_k^{(i)} \sim \mathcal{E}\left(\frac{1}{\tau_{s_k}}\right)$,
where $\tau_j$ is the average dwell time at stop $j$.
\begin{equation}
  \label{eq:start_stop}
  \tilde\delta_k =
  \begin{cases}
    t_k + c_k^{(i)} & t_k - T_{s_k}^{(i)} \geq \gamma, \\
    T_{s_k} + \gamma + c_k^{(i)} & \text{otherwise}.
  \end{cases}
\end{equation}

Each particle then moves on to the next step with $\tilde\delta_k$ time remaining to travel.


\subsection{Setting Particle Speed: will it pass any stops?}
\label{sec:speed_pass}

The first step is to add noise to the system.
This is done by adjusting the speed of each particle.
However, instead of adding random noise, 
we actually want to make use of the speed distribution of the current segment,
which has a mean and variance as estimated by the Kalman filter (\cref{sec:speed}).

To do this, we can use a Metropolis-type step, in which we add noise to each particle,
and then accept the new state with probability $\alpha_i$.
Let $v_{k-1}^{(i)}$ be the initial speed of each particle,
and $v_k^{(i)\star}$ be the proposed speed, then we define
\begin{equation}
  \label{eq:speed_alpha_proposal}
  \alpha_i = \min\left\{
    1, \frac{p\left(v_k^{(i)\star}\,|\, \nu_{s_k}, \xi_{s_k}^2 \right)}{p\left(v_{k-1}^{(i)}\,|\, \nu_{s_k}, \xi_{s_k}^2 \right)}
  \right\},
\end{equation}
where $\nu_{s_k}$ and $\xi_{s_k}^2$ are estimated by the Kalman filter in \cref{sec:speed}.
Then we set $v_k^{(i)} = v_k^{(i)\star}$ with probability $\alpha_i$, otherwise it is set to $v_{k-1}^{(i)}$.


Once the particles have a new speed, we can begin moving them.
First, let $\mathbf{S}^d = \left\{ S_j^d : j = 1, \ldots, M \right\}$ be the distance into trip
of the stops along the route.
We then determine the time until arrival (in seconds) at the next stop, $s_{k-1}^{(i)} + 1$,
\begin{equation}
  \label{eq:time_till_arrival}
  \mathrm{ETA}_{s_{k-1}^{(i)} + 1} = \frac{S^d_{s_{k-1}^{(i)} + 1} - d_{k-1}^{(i)}}{v_k^{(i)}}.
\end{equation}

If the travel time remaining is greater than the ETA, i.e., if $\tilde\delta_k \geq \mathrm{ETA}_{s_{k-1}^{(i)} + 1}$.
then the particle will reach the next stop.

If the particle \emph{does not} reach the next stop, we simply compute the distance into trip for the particle
\begin{equation}
  \label{eq:distance_into_trip_traveled}
  d_k^{(i)} = d_{k-1}^{(i)} + \tilde\delta_k v_k^{(i)}
\end{equation}
and then we are done transitioning the given particle (the remaining steps are skipped).



\subsection{Stop arrival: does it stop?}

At this stage, we assume the bus reaches the next stop.
We therefore set $s_k = s_{k-1} + 1$.
The arrival time is then computed as
\begin{equation}
  \label{eq:arrival_time}
  A_{s_k}^{(i)} = t_{k-1} + \mathrm{ETA}_{s_{k}^{(i)} + 1}
  = t_{k-1} + \frac{S^d_{s_{k}^{(i)} + 1} - d_{k-1}^{(i)}}{v_k^{(i)}}.
\end{equation}

We now determine whether or not the bus stops.
This is simply a Bernoulli random variable, sampled for each particle,
with a probability $\pi_{\ell|r}$, which is the probability
of stopping at stop $\ell$, given that the trip is $r$:
\begin{equation}
  \label{eq:bus_stopped}
  p_{s_k}^{(i)} \sim \mathrm{Bernoilli}\left(\pi_{\ell|r}\right)
\end{equation}
Here, $\pi_{\ell|r}$ can be modelled using a random effects model,
such that
\begin{equation}
  \label{eq:pi_model}
  \mathrm{logit}\left(\pi_{\ell|r}\right) \sim \mathcal{N}\left(\eta_\ell, \sigma_\pi \right).
\end{equation}
More details later.



\subsection{Stop dwell time}

Now we determine the dwell time for each particle,
which is conditionaly on the bus stopping (i.e., $p_j^{(i)} = 1$).
First we define $\gamma$ as the \emph{minimum dwell time},
which encorporates decelleration, opening and closing doors, and acceleration.
$\gamma$ is common for all stops and trips (for now).

The dwell time $\bar t_{s_k}^{(i)}$ is the total dwell time at stop $s_k$,
and is exponentially distributed with expected dwell time $\tau_{\ell|r}$.
Therefore, the total dwell time is
\begin{equation}
  \label{eq:total_dwell_time}
  \gamma + \bar t_{s_k}^{(i)}, \qquad \bar t_{s_k}^{(i)} \sim \mathcal{E}\left(\frac{1}{\tau_{\ell|r}}\right)
\end{equation}


\subsection{Departure time and repeat}

Finally, we compute the departure time.
This combines the above two steps, so that 
\begin{equation}
  \label{eq:departure_time}
  D_{s_k}^{(i)\star} = A_{s_k}^{(i)} + p_{s_k}^{(i)} \left( \gamma + \bar t_{s_k}^{(i)} \right).
\end{equation}

Now, if $D_{s_k}^{(i)\star} < t_k$, then there is still time remaining to continue traveling,
so we set $D_{s_k}^{(i)} = D_{s_k}^{(i)\star}$ and $\tilde\delta_k = t_k - D_{s_k}^{(i)}$,
and then repeat from \cref{sec:speed_pass}.
If $D_{s_k}^{(i)\star} < t_k$, then we leave $D_{s_k}^{(i)}$ unset and end the transition.
In both cases, we set $d_k^{(i)} = S_{s_k}^d$ (i.e., position the particle at the stop).


\section{Route Segment Speed: Kalman Filter}
\label{sec:speed}


The Gaussian assumptions make a lot more sense for speed along a segment;
the only potential cause for concern is the truncation at 0
and $\approx$ 50 or 100 km/h.

The state is
\begin{equation}
  \label{eq:speed_state}
  \bnu_k = 
  \begin{bmatrix}
    \nu_1 \\ \nu_2 \\ \vdots \\ \nu_M
  \end{bmatrix},
\end{equation}
where each $b_j$ is the mean speed along segment $j$.
The state estimate has covariance matrix 
\begin{equation}
  \label{eq:state_cov}
  \Xi_k = 
  \begin{bmatrix}
    \xi_1 & 0 & \cdots & 0 \\
    0 & \xi_2 & \cdots & 0 \\
    \vdots & \vdots & \ddots & \vdots \\
    0 & 0 & \cdots & \xi_M
  \end{bmatrix}
\end{equation}
i.e., it is diagnoal, with all other covariances 0 (for now).

The transition function is simply the identity matrix, $I_M$,
and the system noise is $\bw_k \sim \mathcal{N}(\mathbf{0}, \bQ_k)$,
where $\bQ_k = \Delta_k \sigma_\nu I_M$,
i.e., the average variance in speed per interval of length $\Delta_k$,
and $\sigma_\nu$ is the variation per minute.
The value of $\sigma_\nu$ determines how quickly the speeds can change,
and is therefore rather important\ldots

The transition model is thus
\begin{equation}
  \label{eq:speed_transition_function}
  \bnu_k = I_M \bnu_{k-1} + \bw_k
\end{equation}


For the observations, we simply average over \emph{all} particles that were positioned in
each segment over the past $\Delta_k$ interval,
and also use the sample variance as an estimate for $\bR_k$.

The mean speed in segment $j$ is $\bar z_j$, with variance $\hat v_j^2$.
Therefore, the measurement matrix is also the identity $I_M$.
So the measurement model is
\begin{equation}
  \label{eq:speed_measurement_model}
  \bz_k = I_M \bnu_k + \bv_k,
\end{equation}
where $\bv_k \sim \mathcal{N}(\mathbf{0}, \bR_k)$,
and
\begin{equation}
  \label{eq:speed_covariance_obs}
  \bR_k =
  \begin{bmatrix}
    \hat v_1 & 0 & \cdots & 0 \\
    0 & \hat v_2 & \cdots & 0 \\
    \vdots & \vdots & \ddots & \vdots \\
    0 & 0 & \cdots & \hat v_M
  \end{bmatrix}
\end{equation}

The estimations become fairly simple due to the identity matrices.

\textbf{Predict:}
\begin{align}
  \label{eq:speed_predict1}
  \hat \bnu_{k|k-1} &= \hat\bnu_{k-1|k-1} \\
  \label{eq:speed_predict2}
  \Xi_{k|k-1} &= \Xi_{k-1|k-1} + \bQ_k
\end{align}

\textbf{Update:}
\begin{align}
  \label{eq:speed_update1}
  \tilde\br_k &= \bz_k - \hat\bnu_{k|k-1} \\
  \label{eq:speed_update2}
  \bS_k &= \Xi_{k|k-1} + \bR_k \\
  \label{eq:speed_update3}
  \bK_k &= \Xi_{k|k-1} \bS_k^{-1} \\
  \label{eq:speed_update4}
  \hat\bnu_{k|k} &= \hat\bnu_{k|k-1} + \bK_k \tilde\br_k \\
  \label{eq:speed_update5}
  \Xi_{k|k} &= \left(I_M - \bK_k\right) \Xi_{k|k-1}
\end{align}


In situations where there are no observations in a given segment, 
we just use the previous estimate of $\nu_{j,k-1}$ with a large variance (e.g., $1\times 10^6$).



\section{Arrival Time Prediction}
\label{sec:arrival-time-pred}

The last part consists of predicting arrival times at the remaining stops along the route.
This will be done using a combination of schedule, vehicle state (PF),
and road congestion (KF).


We propose several alternative prediction methods:
\begin{enumerate}
\item \textbf{Schedule}: i.e., disregard all data and just use the static scheduled arrival times.
  This will be our baseline.

\item \textbf{Schedule adherance}: this takes the arrival/departure delay at the most recent stop,
  and applies it to the subsequence stops.

\item \textbf{Current vehicle state}: use the current state (i.e., speed) of the bus and use it to 
  predict the travel time to subsequent stops.
  This will include uncertainty allowing for dwell times at intermediate stops.

\item \textbf{Current traffic state}: instead of using the vehicle state, this method will use the 
  estimated traffic speed for each segment (from the Kalman filter).
  Again, dwell times included as uncertainty.

\item \textbf{Headway}: use the previous bus(es)' travel times to predict arrival times for the current bus.

\item \textbf{Combinations}: compare the accuracy and precision of the above methods,
  and explore combinations of them to predict arrival times.
\end{enumerate}

Methods 2--4 will make explicit use of the particle filter to obtain a distribution of arrival times.
Method 5 could too, but it'll be more complicated because of the historical particle trajectories too.


\subsection{Schedule}
\label{sec:pred-schedule}

Given a vehicle with state $\bX_k$ (see \cref{eq:vehicle_state} on \cpageref{eq:vehicle_state}) 
on a particular trip with $M$ stops,
the scheduled arrival times at stops are $\bS^t = \left\{S_j^t : j = 1, \ldots, M\right\}$.
Then the arrival time prediction at stops $j = s_k + 1, \ldots, M$ are
\begin{equation}
  \label{eq:pred-schedule}
  \hat A_j^{[1]} = S_j^t.
\end{equation}

As a measure of uncertainty, we could use historical data on arrival time variability.
Of course, this could be extended to use \emph{historical} arrival time averages
instead of scheduled, but that data is difficult to obtain.


\subsection{Schedule adherance}
\label{sec:pred-schedule-adherance}

Given a vehicle with state $\bX_k$ on a trip with $M$ stops,
the arrival and departure time delay at stop $j$ are respectively defined as 
\begin{equation}
  \label{eq:arrival-departure-delay}
  \tilde A_j = A_j - S_j^t \qquad \text{and} \qquad \tilde D_j = D_j - S_j^t.
\end{equation}
Note that both use the schedule \emph{arrival} time: this is true for most stops,
except layovers, which have a specific departure time which the bus will wait for.
We will account for these later on;
for now, the routes we work with wont have layovers.


Therefore, the arrival time predictions at stops $j = s_k + 1, \ldots, M$ are
\begin{equation}
  \label{eq:pred-schedule-adherance}
  \hat A_j^{[2]} = 
  \begin{cases}
    S_j^t + \tilde A_{s_k} & \text{if bus is still at stop $s_k$} \\
    S_j^t + \tilde D_{s_k} & \text{if bus has departed stop $s_k$}
  \end{cases}.
\end{equation}
This is done for each particle, generating a distribution of arrival times.

Again, as with the schedule prediction, we could use historical averages in place of scheduled times,
and thus arrival/departure delay would be relative to the average historical arrival/departure time.



\subsection{Current vehicle state}
\label{sec:pred-vehicle-state}

Given a vehicle with state $\bX_k$ on a trip with $M$ stops with distances
$\bS^d = \left\{S_j^d : j = 1, \ldots, M\right\}$ along the route,
the predicted arrival time to each of stops $j = s_k + 1, \ldots, M$ for each particle is given as
\begin{equation}
  \label{eq:pred-vehicle-state}
  \hat A_j^{[3]} = t_k + \frac{S_j^d - d_k}{v_k} + 
  \sum_{\ell = s_k}^{j-1} p_\ell \left( \gamma + \bar t_\ell \right)
\end{equation}
where the summation is simply the accumulated dwell time at intermediate stops
(as defined in \cref{sec:speed_pass}).
Note that if the bus has departed the last stop, then $p_{s_k} = 0$;
otherwise, $p_{s_k} = 1$ and dwell time \emph{remaining} is computed as before.
Also note that to reduce clutter, the superscripts (e.g., $v_k^{(i)}$) have been removed from parameters.

$p_\ell$ and $\bar t_\ell$ will be sampled for each particle, 
based on parameters that are updated using historical data.
An alternative would be to record the mean and variability of historical dwell times,
and use this instead if computational efficiency becomes a problem.


\subsection{Current traffic state}
\label{sec:pred-traffic-state}

Given a vehicle with state $\bX_k$ on a trip with $M$ stops and 
a traffic state represented by $\hat \nu$ and $\Xi$,
we can sample speed $\dot v_j \sim \mathcal{N}(\hat\nu_j, \Xi_{j,j})$ (and therefore travel time) for each particle 
to predict arrival time:
\begin{equation}
  \label{eq:pred-traffic-state}
  \hat A_j^{[4]} = t_k 
  + \underbrace{\frac{S_{s_k + 1}^d - d_k}{\dot v_{s_k}}}_{\text{time to next stop}}
  + \underbrace{\sum_{\ell = s_k + 1}^{j - 1} \frac{S_{\ell + 1}^d - S_\ell^d}{\dot v_\ell}}_{\text{intermediate travel times}}
  + \underbrace{\sum_{\ell = s_k}^{j-1} p_\ell \left( \gamma + \bar t_\ell \right)}_{\text{dwell times}}
\end{equation}


\end{document}
